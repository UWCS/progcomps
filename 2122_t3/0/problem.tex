\addcontentsline{toc}{subsection}{Anagram}
\LARGE \circled{0} \textbf{Example Problem - Anagram} \normalsize

{\itshape A Rag Man.}

This problem is intended to get you familiar with how submission works.

A pair of strings are called anagrams of each other if they share the same letters, but in a different order.
Your task: determine whether given pairs of strings are anagrams.

\vspace{8pt}
\hrule

\textbf{Input}

The first line of the input contains an integer $t$, denoting the number of test cases.

The next $t$ lines of the input consist of two space-separated strings $w_1$ and $w_2$.

\textbf{Constraints}

\begin{itemize}
    \item $1 \leq t \leq 10$
    \item $1 \leq |w_1| = |w_2| \leq 13$
    \item $w_1$ and $w_2$ will be uppercase.
\end{itemize}

\textbf{Output}

$t$ lines each consisting of either "Yes" if the strings are anagrams, or "No" if they're not.

\vspace{8pt}
\hrule

\textbf{Example}

\begin{table}[h]
    \centering
    \begin{tabular}{|p{0.4\linewidth}|p{0.4\linewidth}|}
        \hline
        Input & Output \\
        \hline
        3 \newline BABA ABBA \newline ORANGE CHERRY \newline MOUNTAINEERS ENUMERATIONS & 
        \text{} \newline Yes \newline No \newline Yes \\
        \hline
    \end{tabular}
\end{table}

Note that we have one output line for each of the test cases.

Feel free to give the problem a try yourself! 
However, the main goal of this is for you to understand the submission system, so I've given some example solutions on the next page, and ways to submit them.
These are given for Python and Java, to show the differences.

Once you feel comfortable, let's move onto the first real problem!

\newpage

\textbf{Solutions and Submission}

\lstset{language=python}
\begin{lstlisting}
#!/usr/bin/env python

t = int(input())  # Get number of test cases

for _ in range(t):
    # Get the two words
    w1, w2 = input().split()

    # Convert to sorted character arrays
    s1 = sorted(w1)
    s2 = sorted(w2)

    # Compare the arrays
    if s1 == s2:
        print("Yes")
    else:
        print("No")

\end{lstlisting}

\small
\texttt{./ProgcompCli 0 sol.py} on Mac or Linux, with a shebang. \\
\texttt{.\textbackslash ProgcompCli.exe 0 python.exe -{}-executable-args sol.py} on Windows (remove \texttt{.exe} otherwise).

\vspace{8pt}
\hrule

\lstset{language=java}
\begin{lstlisting}
import java.io.*;
import java.util.*;

public class Solution {
    public static void main(String[] args) {
        // Allows us to read from stdin
        BufferedReader br = new BufferedReader(new InputStreamReader(System.in));

        // Get number of test cases
        int t = 0;
        try {
            t = Integer.parseInt(br.readLine());
        } catch (IOException ex) {}

        String word1, word2;
        for (int i = 0; i < t; i++) { // Repeat for number of test cases
            // Get words
            try {
                String line = br.readLine();
                String[] split = line.split(" ");
                word1 = split[0];
                word2 = split[1];
            } catch (IOException ex) { continue; }
            
            // Sort arrays
            char[] arr1 = word1.toCharArray();
            Arrays.sort(arr1);
            char[] arr2 = word2.toCharArray();
            Arrays.sort(arr2);

            // Compare arrays
            if (Arrays.equals(arr1, arr2)) {
                System.out.println("Yes");
            } else {
                System.out.println("No");
            }
        }
    }
}
\end{lstlisting}

\texttt{.\textbackslash ProgcompCli.exe 0 java -{}-executable-args Solution} on Windows (remove \texttt{.exe} otherwise).
\\ Make sure to compile it first with \texttt{javac}!

\normalsize