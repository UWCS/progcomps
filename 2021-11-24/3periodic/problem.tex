\addcontentsline{toc}{subsection}{Periodic}
\LARGE \textbf{Periodic} \normalsize

Consider the following algorithm:

\begin{enumerate}
    \item Choose a positive, whole number
    \item If it's even, divide by two. Otherwise, multiply by three and add one.
    \item Take the resulting number, and repeat step 2.
\end{enumerate}

In 1937, Lothar Collatz conjectured that no matter the starting number, the result will eventually reach 1. So far, no-one has been able to prove or disprove this.
Interested, you decide to analyse some properties of a given starting number:

\begin{itemize}
    \item \textbf{Strength} - The number of times Step 2 occurs in the algorithm.
    \item \textbf{Reach} - The highest result obtained while applying Step 2.
\end{itemize}

Which numbers within a given interval have the maximum strength and reach?

\vspace{8pt}
\hrule

\textbf{Input}

The first line of the input contains an integer $t$, denoting the number of test cases.

The next $t$ lines of the input contain two space-separated integers $n_1$ and $n_2$. They represent the endpoints of an interval.

\textbf{Constraints}

\begin{itemize}
    \item $1 \leq n_1 \leq n_2 \leq 10^7$
\end{itemize}

\textbf{Output}

Each of the $t$ lines of the output should consist of four space-separated integers:

\begin{itemize}
    \item The smallest starting integer in the interval with the highest strength.
    \item The highest strength.
    \item The smallest starting integer in the interval with the highest reach.
    \item The highest reach.
\end{itemize}

\vspace{8pt}
\hrule

\textbf{Example}

\begin{table}[h]
    \centering
    \begin{tabular}{|p{0.4\linewidth}|p{0.4\linewidth}|}
        \hline
        Input & Output \\
        \hline
        3 \newline 1 6 \newline 9 10 \newline 420 420 & 
        6 8 3 16 \newline 9 19 9 52 \newline 420 40 420 808 \\
        \hline
    \end{tabular}
\end{table}

If our starting number is 6, it takes 8 applications of the second step to reach 1:

\begin{center}
    6 $\rightarrow$ 3 $\rightarrow$ 10 $\rightarrow$ 5 $\rightarrow$ 16 $\rightarrow$ 8 $\rightarrow$ 4 $\rightarrow$ 2 $\rightarrow$ 1
\end{center}

We can also observe that 3 is the smallest number with the highest reach of 16.
The other two cases are similar, with longer chains of numbers.