\addcontentsline{toc}{subsection}{Packing}
\LARGE \textbf{Packing} \normalsize

Welcome to Sahara, the world's largest legally distinct global distributor! 
We have some more boxes for you to pack today, our favourite, dearly valued employee.

If you wish to see your family again, you'll need to figure out how many of these equally shaped rectangular packages will fit together in this other rectangular container. 
Some may call this use of materials 'unnecessary' and 'wasteful', but here, we call it progress.

As we have not yet fully transitioned to the metaverse, it seems no two packages may occupy the same physical space. 
If you require a bathroom break, a free, empty water bottle will be provided for your convenience.

\vspace{8pt}
\hrule

\textbf{Input}

The first line of the input contains an integer $t$, denoting the number of test cases.

The $t$ remaining lines of the input contain four space-separated integers representing the width $w_1$ and height $h_1$ of a package and the width $w_2$ and height $h_2$ of the container.

\textbf{Constraints}

\begin{itemize}
    \item $1 \leq w_1, h_1, w_2, h_2 \leq 10^5$
\end{itemize}

\textbf{Output}

Each of the $t$ output lines should consist of a single integer, representing the maximum number of packages able to fit completely inside the container.

\vspace{8pt}
\hrule

\textbf{Example}

\begin{table}[h]
    \centering
    \begin{tabular}{|p{0.4\linewidth}|p{0.4\linewidth}|}
        \hline
        Input & Output \\
        \hline
        5 \newline 1 1 5 5 \newline 1 2 3 3 \newline 4 5 3 6 \newline 2 10 6 9 \newline 8 41 40 40 & 
        25 \newline 4 \newline 0 \newline 1 \newline 2 \\
        \hline
    \end{tabular}
\end{table}

\begin{itemize}
    \item 25 packages of size 1 x 1 can fit in a container of size 5 x 5.

    \item We can fit four 1 x 2 packages in a 3 x 3 container after rotating two of them.
    
    \item A 4 x 5 package has area 20, the container has area 3 x 6 = 18.
    
    \item It is possible to rotate a single 2 x 10 package to fit in a 6 x 9 container.
    
    \item It is possible to rotate two 8 x 41 packages to fit inside a 40 x 40 container.
\end{itemize}

\vspace{8pt}
\hrule

\textbf{Employee Note}

Packages should only be rotated to two distinct angles per container. We don't have time for some of your crazy mathematical technicalities in the real world!