\togglefalse{IsScore}
\problem{2}{Industry Standard}{Keegan R}
{Job Requirements: 10 years experience in UWPL}

I know what you're thinking. You're sick of all these dumb, annoying, overengineered programming languages, aren't you?
Pointers? Classes? Types? Functions? Conditions? Too complicated.
That's why we're developing the tool to drive future innovation - UWPL.

The philosophy of UWPL is minimality. We ensure this with the two principles:
\begin{enumerate}
    \item \textit{"Control Variability"} -
    The integer is all.
    All begins at zero.
    All begins nameless.
    
    \item \textit{"Control Complexity"} - 
    We increment, and we repeat.
    That is all.
\end{enumerate}

\begin{table}[h]
    \centering
    \begin{tabular}{l p{3cm} l}
        \Huge \texttt{eggs\textcolor{uwcsblue}{++}} & & \Huge \texttt{\textcolor{uwcsred}{repeat} 3} \\
        \Huge \texttt{flour\textcolor{uwcsblue}{++}} & & \Huge \texttt{\textcolor{uwcsred}{$\lbrace$}} \\
        \Huge \texttt{sugar\textcolor{uwcsblue}{++}} & & \Huge \texttt{apples\textcolor{uwcsblue}{++}} \\
        \Huge \texttt{flour\textcolor{uwcsblue}{++}} & & \Huge \texttt{apples\textcolor{uwcsblue}{++}} \\
        \Huge \texttt{eggs\textcolor{uwcsblue}{++}} & & \Huge \texttt{olives\textcolor{uwcsblue}{++}} \\
        \Huge \texttt{flour\textcolor{uwcsblue}{++}} & & \Huge \texttt{\textcolor{uwcsred}{$\rbrace$}} \\
    \end{tabular}
    \caption*{Examples of UWPL scripts. Elegant, aren't they?}
\end{table}

At the end of each program, the non-zero variable values should be output.
Unfortunately, UWPL has not yet been widely adopted - so you'll need to create an interpreter for that.

\vspace{8pt}
\hrule

\textbf{Input}

The first line of the input contains an integer $n$, denoting the number of test cases.

The next $n$ groups of lines consist of:
\begin{itemize}
    \item An integer $m$, the length of the upcoming UPWL script
    \item A valid UWPL script with $m$ lines. Each line could be:
    \begin{itemize}
        \item An increment statement on one of $k$ variable names $v_i$, for $i \in \{1, \dots, k\}$.
        \item Part of a "repeat $a$" statement. 
        The scope of the repeat is denoted by an opening brace $\lbrace$ on the following line, and a closing brace $\rbrace$ on a later line.
    \end{itemize}
\end{itemize}

\textbf{Constraints}
\begin{equation*}
    1 \leq n \leq 10^2 \qquad
    1 \leq m \leq 10^4 \qquad
    1 \leq a \leq 10^2 \qquad
    1 \leq k \leq 10^2 \qquad
    \forall i \in \{1, \dots, k\} \quad 1 \leq |v_i| \leq 5
\end{equation*}

The final value of any variable $v_i$ cannot exceed $10^6$. All inputs are valid UWPL scripts.

\textbf{Output}

$n$ pairs of lines. Each pair represents the end state of a UWPL script:
\begin{enumerate}
    \item A space-separated list of variable names $v_i$, for $i \in \lbrace 1, \dots, k \rbrace$.
    \item A space-separated list of integers corresponding to the final value of each $v_i$.
\end{enumerate}

Note: Your output order of variable names does not matter.

\vspace{8pt}
\hrule

\newpage

\textbf{Example}

\begin{table}[h]
    \centering
    \begin{tabular}{|p{0.4\linewidth}|p{0.4\linewidth}|}
        \hline
        Input & Output \\ \hline
        \texttt{3\newline
        \textbf{5}\newline
        two++\newline
        three++\newline
        three++\newline
        two++\newline
        two++\newline
        \textbf{8}\newline
        a++\newline
        repeat 5\newline
        $\lbrace$\newline
        b++\newline
        c++\newline
        b++\newline
        $\rbrace$\newline
        c++\newline
        \textbf{17}\newline
        cat++\newline
        repeat 7\newline
        $\lbrace$\newline
        dog++\newline
        repeat 10\newline
        $\lbrace$\newline
        cat++\newline
        snake++\newline
        $\rbrace$\newline
        duck++\newline
        repeat 11\newline
        $\lbrace$\newline
        snake++\newline
        duck++\newline
        $\rbrace$\newline
        $\rbrace$\newline
        duck++}
        & 
        \texttt{two three\newline
        3 2\newline
        a c b\newline
        1 6 10\newline
        cat dog snake duck\newline
        71 7 147 85\newline
        }\\ \hline
    \end{tabular}
\end{table}

A few notes about these test cases:

\begin{enumerate}
    \item The variable \texttt{two} is incremented 3 times, and \texttt{three} is incremented 2 times.
    \item This example shows off the basic usage and scope of the repeat block.
    \item Repeat statements can be arbitrarily nested, so be careful with your scope!
\end{enumerate}