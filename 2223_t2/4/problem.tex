\toggletrue{IsScore}
\problem{4}{Third Principle}{Keegan R}
{The road to hell is paved with good intentions}

Remember our hit new programming language, UWPL? I bet you thought you were done with that.
It turns out there was a third and final principle all along:
\begin{enumerate}
    \item[3.] \textit{"Control Space"} - 
    Reject line non-minimality. Less lines, more UWPL.
\end{enumerate}

In other words, the UWPL compiler simply rejects any program with unneccessary lines.
Or at least it will, when we get around to building it.
In the meantime, ensure that your UWPL programs adhere to this standard as much as possible.
Clean code. A clean mind.

% diagram of long code, arrow, shorter code
\begin{figure}[h]
    \centering
    \begin{tikzpicture}
        \node at (-4.5, 0) {
            \begin{tabular}{l}
                \Huge \texttt{cheese\textcolor{uwcsblue}{++}} \\
                \Huge \texttt{\textcolor{uwcsred}{repeat} 3} \\
                \Huge \texttt{\textcolor{uwcsred}{$\lbrace$}} \\
                \Huge \texttt{tomato\textcolor{uwcsblue}{++}} \\
                \Huge \texttt{cheese\textcolor{uwcsblue}{++}} \\
                \Huge \texttt{tomato\textcolor{uwcsblue}{++}} \\
                \Huge \texttt{\textcolor{uwcsred}{$\rbrace$}} \\
                \Huge \texttt{cheese\textcolor{uwcsblue}{++}} \\
            \end{tabular}
        };
        \node[single arrow, fill=uwcsblue, minimum width = 2cm, minimum height = 3.5cm] {\bfseries \Large \textcolor{white}{Minimal}};
        \node at (4.5, 0) {
            \begin{tabular}{l}
                \Huge \texttt{\textcolor{uwcsred}{repeat} 5} \\
                \Huge \texttt{\textcolor{uwcsred}{$\lbrace$}} \\
                \Huge \texttt{tomato\textcolor{uwcsblue}{++}} \\
                \Huge \texttt{cheese\textcolor{uwcsblue}{++}} \\
                \Huge \texttt{\textcolor{uwcsred}{$\rbrace$}} \\
                \Huge \texttt{tomato\textcolor{uwcsblue}{++}} \\
            \end{tabular}
        };
    \end{tikzpicture}
\end{figure}

Given a specified end state of variables, construct a UWPL program achieving it. 
\\ You will be scored on how well your program adheres to the third principle.

\vspace{8pt}
\hrule

\textbf{Input}

The first line of the input contains an integer $n$, denoting the number of test cases.

The next $n$ groups of lines consist of:
\begin{itemize}
    \item A line containing a single integer $k$, the number of non-zero variables.
    \item A line of $k$ space-separated strings, $s_1, s_2, \dots, s_k$.
    \item A line of $k$ space-separated integers, $v_1, v_2, \dots, v_k$.
\end{itemize}

The pair $(s_i, v_i)$ represents the name and final value of the $i$th variable.

\textbf{Constraints}
\begin{equation*}
    1 \leq n \leq 10^2 \qquad
    1 \leq k \leq 10^3 \qquad
    \forall i \in \{1, \dots, k\} \quad 1 \leq |s_i| \leq 10, \quad 1 \leq v_i \leq 10^6
\end{equation*}
\textbf{Output}

$n$ groups of lines, where:
\begin{itemize}
    \item The first line of each group is an integer $m$, the length of your UWPL script.
    \item The next $m$ lines should be a UWPL script yielding the required end state.
\end{itemize}

\vspace{8pt}
\hrule

\textbf{Scoring}

For each test case, if the output is invalid, get \textbf{-20} points. Otherwise, gain $4k - m$ points.

\vspace{8pt}
\hrule

\newpage

\textbf{Example}

\begin{table}[h]
    \centering
    \begin{tabular}{|p{0.4\linewidth}|p{0.4\linewidth}|}
        \hline
        Input & Output \\
        \hline
        \texttt{4\newline 
        \textbf{2}\newline 
        xx yy\newline 
        3 3\newline 
        \textbf{3}\newline 
        xx yy zz\newline
        35 90 55\newline
        \textbf{1}\newline 
        xx\newline
        3\newline
        \textbf{2}\newline
        xx yy\newline
        6 3}
        & \texttt{\textbf{5}\newline 
        repeat 3\newline
        $\lbrace$\newline 
        xx++\newline 
        yy++\newline 
        $\rbrace$\newline
        \textbf{12}\newline 
        repeat 35\newline
        $\lbrace$\newline 
        xx++\newline 
        $\rbrace$\newline 
        repeat 55\newline
        $\lbrace$\newline 
        zz++\newline 
        $\rbrace$\newline 
        repeat 90\newline
        $\lbrace$\newline 
        yy++\newline 
        $\rbrace$\newline
        \textbf{3}\newline 
        xx++\newline
        xx++\newline 
        xx++\newline
        \textbf{6}\newline 
        repeat 3\newline 
        $\lbrace$\newline 
        xx++\newline 
        xx++\newline
        yy++\newline
        $\rbrace$\newline} \\
        \hline
    \end{tabular}
\end{table}

A few notes about these test cases:

\begin{enumerate}
    \item Variables with the same value may be worth merging into the same repeat.
    \item Sometimes, the best way is to handle them separately.
    \item Repeats take up a lot of space, so might not be necessary.
    \item It may be more compact to use the same variable twice within a repeat.
\end{enumerate}

The total score for this submission would be \textbf{+6}.