\toggletrue{IsScore}
\problem{5}{Tic-Tac-No}{Keegan R}
{My final problem.}

Tic-tac-toe is a relatively simple game. To win, you need to form a line/chain of 3 squares (horizontally, vertically, or diagonally) on a $3 \times 3$ grid.
This problem is less concerned with the game, and more with those chains.
How many grid squares would we need to block to prevent any 3-chain from being formed?
The answer is 3: top left, middle, bottom right.

Given an $n \times n$ grid, block a subset of grid squares to prevent any $k$-chain from being formed. 
Use less squares, gain more points.
We discussed the case where $n=3, k=3$.

% Picture showing non-doomed and doomed queens
\begin{figure}[h]
    \centering
    \begin{tikzpicture}[scale=0.85]
        % Grid 1
        \begin{scope}[shift={(0, 2.5)}]
            \draw[step=1cm, black, thin] (0, 0) grid (3, 3);
            
            \filldraw[uwcsred] (0, 0) rectangle ++(1, 1);
            \filldraw[uwcsred] (1, 1) rectangle ++(1, 1);
            \filldraw[uwcsred] (2, 2) rectangle ++(1, 1);

            \node at (1.5, -0.5) {\Large $n=3, k=3$};
        \end{scope}

        % Grid 2
        \begin{scope}[shift={(4, 1)}]

            \draw[step=1cm, black, thin] (0, 0) grid (6, 6);
        
            \filldraw[uwcsred] (2, 0) rectangle ++(2, 1);
            \filldraw[uwcsred] (2, 5) rectangle ++(2, 1);
            \filldraw[uwcsred] (0, 2) rectangle ++(1, 2);
            \filldraw[uwcsred] (5, 2) rectangle ++(1, 2);
            
            \filldraw[uwcsred] (1, 1) rectangle ++(1, 1);
            \filldraw[uwcsred] (4, 1) rectangle ++(1, 1);
            \filldraw[uwcsred] (1, 4) rectangle ++(1, 1);
            \filldraw[uwcsred] (4, 4) rectangle ++(1, 1);

            \filldraw[uwcsred] (2, 2) rectangle ++(2, 2);
    
            \node at (3, -0.5) {\Large $n=6, k=3$};
        \end{scope}

        % Grid 3
        \begin{scope}[shift={(11, 0)}]

            \draw[step=1cm, black, thin] (0, 0) grid (8, 8);
        
            \filldraw[uwcsred] (0, 0) rectangle ++(1, 1);
            \filldraw[uwcsred] (3, 0) rectangle ++(1, 1);
            \filldraw[uwcsred] (4, 0) rectangle ++(1, 1);
            \filldraw[uwcsred] (2, 1) rectangle ++(1, 1);
            \filldraw[uwcsred] (6, 1) rectangle ++(1, 1);

            \filldraw[uwcsred] (3, 2) rectangle ++(1, 3);
            \filldraw[uwcsred] (7, 2) rectangle ++(1, 1);

            \filldraw[uwcsred] (1, 3) rectangle ++(1, 1);
            \filldraw[uwcsred] (5, 3) rectangle ++(1, 1);

            \filldraw[uwcsred] (0, 4) rectangle ++(1, 1);
            \filldraw[uwcsred] (4, 4) rectangle ++(1, 1);
            \filldraw[uwcsred] (7, 4) rectangle ++(1, 1);

            \filldraw[uwcsred] (2, 5) rectangle ++(1, 1);
            \filldraw[uwcsred] (6, 5) rectangle ++(1, 1);
            \filldraw[uwcsred] (3, 6) rectangle ++(2, 1);

            \filldraw[uwcsred] (1, 7) rectangle ++(1, 1);
            \filldraw[uwcsred] (5, 7) rectangle ++(1, 1);

            \node at (4, -0.5) {\Large $n=8, k=4$};
        \end{scope}

    \end{tikzpicture}
\end{figure}
Some values of $n$ and $k$ are easier to work with than others.
Remember: block $k$-in-a-row.

\vspace{8pt}
\hrule

\textbf{Input}

The first line of the input contains an integer $n$, denoting the number of test cases.

Each of the $t$ test cases consists of two space-separated integers:
\begin{itemize}
    \item $n$ is the side length of the grid.
    \item $k$ is the block length. No unblocked chain of this size or higher may be formed.
\end{itemize}

\textbf{Constraints}
\begin{equation*}
    1 \leq t \leq 100 \qquad
    1 \leq k \leq n \leq 10^3
\end{equation*}

\textbf{Output}

For each of the $t$ test cases, output 2 lines representing $b$ coordinates:
\begin{enumerate}
    \item A space-separated list of $x$-coordinates $x_1, x_2, \dots, x_b$, with $1 \leq x_i \leq n$
    \item A space-separated list of $y$-coordinates $y_1, y_2, \dots, y_b$, with $1 \leq y_i \leq n$
\end{enumerate}

\vspace{8pt}
\hrule

\textbf{Scoring}

For each test case, if the output is invalid, get \textbf{-20} points. Otherwise, you gain:
\begin{equation*}
    2n \left\lfloor \frac{n}{k} \right\rfloor - \left\lfloor \frac{n}{k} \right\rfloor^2 - b
\end{equation*}

\vspace{8pt}
\hrule

\newpage

\textbf{Example}

\begin{table}[h]
    \centering
    \begin{tabular}{|p{0.2\linewidth}|p{0.6\linewidth}|}
        \hline
        Input & Output \\ \hline
        \texttt{3\newline 
        3 3\newline
        6 3\newline
        8 4\newline}
        &
        \texttt{1 2 3\newline
        1 2 3\newline
        1 1 2 2 3 3 3 3 4 4 4 4 5 5 6 6\newline
        3 4 2 5 1 3 4 6 1 3 4 6 2 5 3 4\newline
        1 1 2 2 3 3 4 4 4 4 4 5 5 5 6 6 7 7 8 8\newline
        1 5 4 8 2 6 1 3 4 5 7 1 5 7 4 8 2 6 3 5\newline}
        \\ \hline
    \end{tabular}
\end{table}

The visualised version of these test cases is above.

Best of luck!
