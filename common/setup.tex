\section*{Setup}

To take part, you'll need to download a program we made to interact with your script. 
It will send input data, recieve the output data, and give back a mark or score.

Donwload the ProgcompCLI here: \url{https://github.com/Jlobblet/ProgcompMarker/releases}

Select the download that's appropriate to your operating system and architecture. 
On a DCS machine, this would be \texttt{linux-x64}. 
On most Windows machines, this is likely \texttt{win-64}. 
You can choose whether to download it as a zip file (or tarball), which you'll need to extract.

In the extracted folder, you'll find an exectuable called \texttt{ProgcompCli}. 
This will have the \texttt{.exe} extension if you're on windows. Run it from the command line with no arguments.

You should now be prompted for first-time setup. Enter \texttt{https://progcomp.uwcs.co.uk} for the endpoint, and pick a username which you think will identify you.

\subsection*{Using the ProgcompCLI}

To submit a solution, you'll need to provide at least two arguments: the problem number in this document such as \circled{0}, and the path to the executable script you made to solve it.

By default, data will be sent from the server via standard input (\texttt{stdin}), and your standard output (\texttt{stdout}) will be redirected to the server to be marked.
However, this can be changed! Check out the command options for more details.

If your language can directly output an executable, then you should be able to pass it as the argument.
However, most languages do not do this by default.

\texttt{./ProgcompCli 0 Solution.exe}

If you're using an interpreted language and a UNIX-based system (Mac, Linux), you might want to add a \href{https://en.wikipedia.org/wiki/Shebang_(Unix)}{\em shebang} as the first line of your script, allowing it to be executed without any arguments.
For Python, this is likely to be \verb|#!/usr/bin/env python|. 
Otherwise, use the \texttt{executable-args} flag to pass in multiple arguments. Here is a Windows Java example:

\texttt{.\textbackslash ProgcompCli.exe 0 java -{}-executable-args Solution}