\addcontentsline{toc}{subsection}{Polynomial}
\LARGE \textbf{Problem 4 - Polynomial} \normalsize

In mathematics, a polynomial equation can be written in the form

\begin{equation*}
    a_n x^n + a_{n-1} x^{n-1} + \dots + a_2 x^2 + a_1 x + a_0 = 0
\end{equation*}

With all of the $a_i$ integers. A polynomial of degree $n$ has $n$ solutions.

Given all the (not necessarily distinct) integer solutions to a polynomial equation, you wish to output it in this expanded form.

\vspace{8pt}
\hrule

\textbf{Input}

The first line of the input contains an integer $t$, denoting the number of test cases.

Each test case consists of two lines. The first is $s$, the number of solutions to the equation. 
The second line consists of $s$ space separated integers $r_1, r_2, \dots, r_s$.

\textbf{Constraints}

\begin{itemize}
    \item $1 \leq s \leq 25$
    \item $-10 \leq r_i \leq 10$ for all $i \in \{ 1, \dots, s\}$
\end{itemize}

\textbf{Output}

$t$ lines each consisting of an expanded polynomial equation in descending powers. 
The first cooefficient should be positive, with space-separated signs and terms.

As per mathematical convention, note that:

\begin{itemize}
    \item If the coefficient of a power is 0, the term is excluded.
    \item If the coefficient of a power is 1 or -1, the coefficient is removed.
\end{itemize}

See the example output for more information.

\vspace{8pt}
\hrule

\textbf{Example}

\begin{table}[h]
    \centering
    \begin{tabular}{|p{0.4\linewidth}|p{0.4\linewidth}|}
        \hline
        Input & Output \\
        \hline
        3 \newline 2 \newline 1 1 \newline 4 \newline 2 -1 1 -2 \newline 6 \newline 0 0 0 0 0 -1 & 
        x\^{}2 - 2x + 1 = 0 \newline x\^{}4 - 5x\^{}2 + 4 = 0 \newline x\^{}6 + x\^{}5 = 0 \\
        \hline
    \end{tabular}
\end{table}

To calculate the output, we expand the following three equations:

\begin{itemize}
    \item $(x - 1)(x - 1) = 0$
    \item $(x - 2)(x + 1)(x - 1)(x + 2) = 0$
    \item $x\cdot x \cdot x \cdot x \cdot x \cdot (x + 1) = 0$
\end{itemize}