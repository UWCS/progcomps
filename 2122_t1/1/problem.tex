\addcontentsline{toc}{subsection}{Persistence}
\LARGE \textbf{Problem 1 - Persistence} \normalsize

The additive persistence of an integer is the smallest number of times you need to repeatedly sum its digits, in order to reduce it to a single digit.

\begin{center}
    \Huge 586 $\rightarrow$ 19 $\rightarrow$ 10 $\rightarrow$ 1 \normalsize
\end{center}


Given a positive integer, you want to determine the additive persistence of it.

\vspace{8pt}
\hrule

\textbf{Input}

The first line of the input contains an integer $t$, denoting the number of test cases.

The next $t$ lines of the input consist of a single integer $n$.

\textbf{Constraints}

\begin{itemize}
    \item $1 \leq l \leq 10^5$
    \item $1 \leq n \leq 10^9$
\end{itemize}

\textbf{Output}

$t$ lines each consisting of a single integer, the additive persistence of $n$.

\vspace{8pt}
\hrule

\textbf{Example}

\begin{table}[h]
    \centering
    \begin{tabular}{|p{0.4\linewidth}|p{0.4\linewidth}|}
        \hline
        Input & Output \\
        \hline
        3 \newline 39 \newline 5 \newline 420 & 
        2 \newline 0 \newline 1 \\
        \hline
    \end{tabular}
\end{table}

The first line of the input specifies there are 3 test cases:

\begin{itemize}
    \item For the input $39$, we calculate $3 + 9 = 12$, then $1 + 2 = 3$, so we output $2$.
    \item $5$ is already a single digit, so no digit sums are necessary, we output $0$.
    \item For $420$ we only calculate $4 + 2 + 0 = 6$, so we output $1$.
\end{itemize}