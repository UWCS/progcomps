\addcontentsline{toc}{subsection}{Shift}
\LARGE \circled{1} \textbf{Introductory Problem - Shift} \normalsize

{\itshape I'm sure they'll figure out a use for this}

For computers to automatically check spelling, 
they need to guess which word we were attempting to write.
One method is to find a notion of 'distance' between what was typed against a word dictionary, and suggest the lowest-scoring options.
These notions of distance are known as metrics, with the most popular being the Levenshtein word metric.

Let's make our own word metric! It's called \texttt{SHIFT}\footnote{in conversation, note that the 'F' is pronounced silently}, and consists of two operations:

\begin{enumerate}
    \item Shift a letter a position up or down the alphabet (with wraparound)
    \item Shift a letter on either edge of a word to the other side.
\end{enumerate}

The \texttt{SHIFT} distance is defined as the minimum number of operations we need to make to get from one word to another.

Given two words of the same length, what is the \texttt{SHIFT} distance between them?

\vspace{8pt}
\hrule

\textbf{Input}

The first line of the input contains an integer $t$, denoting the number of test cases.

The next $t$ lines of the input consist of two space-separated strings $w_1$ and $w_2$.

\textbf{Constraints}

\begin{itemize}
    \item $1 \leq t \leq 10000$
    \item $1 \leq |w_1| = |w_2| \leq 13$
    \item $w_1$ and $w_2$ will be uppercase.
\end{itemize}

\textbf{Output}

$t$ lines each consisting of a single integer, the \texttt{SHIFT} distance between $w_1$ and $w_2$.

\vspace{8pt}
\hrule

\textbf{Example}

\begin{table}[h]
    \centering
    \begin{tabular}{|p{0.4\linewidth}|p{0.4\linewidth}|}
        \hline
        Input & Output \\
        \hline
        5 \newline READ REED \newline ZERO HERO \newline PAID FEED \newline WORDS SWORD \newline REDDIT CRINGE & 
        \text{} \newline 4 \newline 8 \newline 18 \newline 1 \newline 31 \\
        \hline
    \end{tabular}
\end{table}

Let's walk through these 5 test cases:

\begin{enumerate}
    \item We shift the A in READ down by 4 to get to REED.
    \item We shift the Z in ZERO down by 8 to get to HERO.
    \item We shift P up by 10, A down by 4, and I up by 4 to get to FEED.
    \item We shift WORDS to the right by 1, and it immediately becomes SWORD.
    \item We shift to the left by 3 to get DITRED, then perform 28 further up/down shifts.
\end{enumerate}