\addcontentsline{toc}{subsection}{Printing}
\LARGE \textbf{Problem 2 - Printing} \normalsize

{\itshape I wish it worked like this in real life}

Alexa has some 3D printers, but isn't sure what to do with them at the moment.
Suddenly, a flash of inspiration hits:

\begin{center}
    \textit{What if I printed more 3D printers with my 3D printer?}
\end{center}

She gathers the needed materials and sets the process running. 
It takes an entire two days for each printer to print a new printer (the same model of course), 
and all newly printed printers will be set up and ready to print after a single day.
What do you do with a new 3D printer? Why, print more printers, of course!

You can assume that, in her bid for global market domination, 
Alexa always has the resources to build and maintain these printers, 
and that as soon as a printer finishes printing, it's immediately set to print again. 
Machines don't get days off.

It's your job to track the total number of printers on a given day.

\vspace{8pt}
\hrule

\textbf{Input}

The first line of the input contains an integer $t$, denoting the number of test cases.

The next $t$ lines of the input contain two space-separated integers:
\begin{itemize}
    \item The first is $p$, the number of printers Alexa starts with.
    \item The second is $d$, the number of days Alexa lets the printing process run for.
\end{itemize}

\textbf{Constraints}

\begin{itemize}
    \item $0 \leq p \leq 10^4$
    \item $0 \leq d \leq 100$
\end{itemize}

\textbf{Output}

$t$ lines each consisting of a single integer, the number of printers after $d$ days.

This will not exceed the maximum value of a 64-bit signed integer.

\vspace{8pt}
\hrule

\textbf{Example}

\begin{table}[h]
    \centering
    \begin{tabular}{|p{0.4\linewidth}|p{0.4\linewidth}|}
        \hline
        Input & Output \\
        \hline
        3 \newline 1 5 \newline 4 13 \newline 0 100 & 
        \text{} \newline 4 \newline 148 \newline 0 \\
        \hline
    \end{tabular}
\end{table}

There are 3 test cases:

\begin{enumerate}
    \item Starting with 1 printer, after 5 days we'll have 4 printers (see \href{https://uwcs.co.uk/media/images/firefox_XuzeaIzMRs.original.png}{image}).
    \item Starting with 4 printers, after 13 days we'll have 148 printers.
    \item Starting with 0 printers, no matter how many days pass we will still have 0.
\end{enumerate}