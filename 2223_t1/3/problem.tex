\addcontentsline{toc}{subsection}{My Server Rules}
\LARGE \circled{3} \textbf{My Server Rules} \normalsize

{\itshape How am I even meant to follow these?}

As an assistant to the supreme leader of a newly created Discord server, you have been tasked with enforcing the rules with an iron fist.
Unfortunately for you, your wise and benevolent leader got a little bit carried away while writing them. 
Not only are there a ridiculous number of rules, but it looks like the ultimate authority on this extremely small corner of the internet likes changing their mind - some contradict or negate each other!

You need to figure out which rules are actually in effect, and remove any redundancies.

\vspace{8pt}
\hrule

\textbf{Input}

The first line of the input contains an integer $n$, denoting the number of test cases. \\
There will then be $n$ of the following:
\begin{itemize}
    \item A line containing a single integer $k$, the number of rules.
    \item $k$ lines consisting of two space-separated variables. 
    Firstly, a unique integer $1 \leq i \leq k$.
    Secondly, a string $r_i$ describing the rule. There are three types of rules:
    \begin{itemize}
        \item (Ignore) The format \texttt{ignore rule }$j$, for some $1 \leq j \leq k$.
        \item (Negative) The format \texttt{don't }$r'$, where $r'$ is in the form of a Positive rule.
        \item (Positive) Any other format.
    \end{itemize}
\end{itemize}

\textbf{Constraints}
\begin{equation*}
    1 \leq n \leq 10^3 \qquad 
    1 \leq k \leq 10^5 \qquad 
    1 \leq |r_k| \leq 100
\end{equation*}

\textbf{Output}

$n$ lines of space-separated integers in \textit{any} order, the rules in force after this procedure:
\begin{enumerate}
    \item Apply Ignore rules, which prevent another rule from taking effect.
    \item Merge duplicate rules, but keep the lowest rule number between the copies.
    \item Delete the Positive and Negative rules which contradict each other.
    \item Wipe away any collection of rules creating a logical paradox.
\end{enumerate}
Earlier steps in the procedure take priority over later ones, where multiple can be applied.
If after applying the procedure, no rules remain, write the output line \texttt{"No Rules"} instead.

\vspace{8pt}
\hrule

\textbf{Example}

Continued on next page \dots

\newpage

\begin{table}[h]
    \centering
    \begin{tabular}{|p{0.4\linewidth}|p{0.4\linewidth}|}
        \hline
        Input & Output \\
        \hline
        4 
        \newline \textbf{3}
        \newline 1 touch grass
        \newline 2 don't touch grass
        \newline 3 ignore rule 2
        \newline \textbf{3}
        \newline 1 don't bow to our supreme leader
        \newline 2 bow to our supreme leader
        \newline 3 don't bow to our supreme leader
        \newline \textbf{5}
        \newline 1 ignore rule 4
        \newline 2 don't spam
        \newline 3 don't post cringe
        \newline 4 ignore rule 2
        \newline 5 have a brain
        \newline \textbf{5} 
        \newline 1 ignore rule 3
        \newline 2 don't put memes in general
        \newline 3 ignore rule 1
        \newline 4 ignore rule 4
        \newline 5 don't put memes in general
        &  
        1 \newline No Rules \newline 3 5 2 \newline 2 \\
        \hline
    \end{tabular}
\end{table}

Let's walk through these four test cases:

\begin{enumerate}
    \item Rules 1 and 2 contradict, but rule 2 is first ignored by rule 3, which then disappears. Therefore, only rule 1 is necessary.
    \item Rules 1 and 3 are merged, but we keep the number 1. Then rules 1 and 2 contradict, so there are none remaining.
    \item Rule 4 ignores rule 2, but rule 1 ignores rule 4, so rules 2, 3, 5 are in effect.
    They can be output in any order.
    \item Rule 1 and 3 together create a logical paradox. 
    Rule 4 creates a paradox with itself.
    Rules 2 and 5 are merged into rule 2, and this is the only rule remaining.
    
\end{enumerate}