\documentclass[a4paper,11pt,parskip=half-]{scrartcl} 
% scrartcl for Montserrat

\usepackage[a4paper,margin=2cm,marginparwidth=1.75cm]{geometry}

\usepackage[english]{babel}
\usepackage{array}
\usepackage{amsmath}
\usepackage[table]{xcolor}
\usepackage[colorlinks=true, allcolors=blue]{hyperref}
\usepackage{graphicx}

\usepackage[defaultfam,tabular]{montserrat}
\usepackage[scaled]{beramono}
\usepackage[T1]{fontenc}

\usepackage{wrapfig}  % for top-right logo

\usepackage[most]{tcolorbox}
\usepackage{tikz}

\usepackage{caption}
\usepackage{subcaption}

% Custom colours
\definecolor{uwcsblue}{RGB}{58,125,255}
\definecolor{uwcsred}{RGB}{238,79,79}
\definecolor{uwcsyellow}{RGB}{255,199,0}

% For correct/incorrect
\newcommand*\circled[1]{
    \tikz[baseline=(char.base)]{
    \node[shape=circle,inner sep=2pt,fill=uwcsblue] 
    (char) {\textcolor{white}{\textbf{#1}}};
    }
}

% For score-based
\newcommand*\squared[1]{
    \tikz[baseline=(char.base)]{
    \node[shape=rectangle,inner sep=4pt,fill=uwcsred] 
    (char) {\textcolor{white}{\textbf{#1}}};}
}

\usepackage{etoolbox} % for conditional execution
\newtoggle{IsScore}

\newcommand*\problem[4]{
    \addcontentsline{toc}{subsection}{#1}
    {\LARGE 
    \iftoggle{IsScore}
    {
        \squared{#1}
    }
    {
        \circled{#1}
    } 
    \textbf{#2}
    } 
    \hfill
    {\footnotesize Created by: \itshape #3}

    {\itshape #4}
}

\usetikzlibrary{shapes.arrows}

\graphicspath{{../common/}}
\pagestyle{empty}
\setkomafont{section}{\usefont{T1}{fvs}{b}{n}\Large}

\renewcommand{\arraystretch}{1.5}

% Custom logo footer
\usepackage{fancyhdr}

\pagestyle{fancy}
\fancyhf{}
\renewcommand{\headrulewidth}{0pt}
\fancyfoot[EOCF]{\includegraphics[scale=0.05]{logo_wide.png}}
\fancyfoot[R]{\thepage}


\begin{document}

\begin{wrapfigure}{r}{0.2\textwidth}
    \vspace{-10pt} % Move in line with text
    \includegraphics[width=0.9\linewidth]{shield.png}
    \vspace{-100pt} % Prevent text below from moving out of the way
\end{wrapfigure}

{
    \Huge \bfseries UWCS Challenge
}

{
    \Large A friendlier programming competition
}

\section*{Introduction}

Welcome to our internal progcomp!
There will be problems. There will be prizes! And of course, there will be pizza.
This document contains specifications for the 5 problems.

For each problem, you will need to process your solution on various input datasets, then submit the outputs (and your code) for scoring.
This is all handled by our website:
\begin{center}
    \huge \href{https://challenge.uwcs.co.uk}{challenge.uwcs.co.uk}
\end{center}

There are 100 (+25) points available for each problem. 
The base 100 points are divided across each input file. 
For example - if there are 5 input files, each is worth up to 20 points. 
Each input file has multiple test cases, so you can get partial marks.
Fully complete a problem and get a bonus 25 points - a nice incentive for finishing it off!

Later problems may be score-based, so won't have a designated solution.
For these, points are awarded by ranking teams on their score for each input dataset, and scaling that from 0 to 100.
To be ranked, your score must be non-negative.

The first 3 problems will be released simultaneously.
The rest will be staggered. \\ 
Good luck, and have fun!

\vspace{16pt}
\hrule
\vspace{16pt}

\togglefalse{IsScore}
\problem{0}{Reverse}{N/A}
{esreveR}

This sample problem is about reversing strings. \textbf{It does not count towards your score.}

\vspace{8pt}
\hrule

\textbf{Input}

The first line of the input contains an integer $n$ - the number of test cases. \\ 
The next $n$ input lines are strings ($s_1, s_2, \dots, s_n$) of only lowercase characters.

\textbf{Constraints}
\begin{equation*}
    1 \leq n \leq 10^2 \qquad 
    \forall i \in \{1, \dots, n\}, \quad 1 \leq |s_i| \leq 10^2
\end{equation*}

\textbf{Output}

$n$ lines, with line $i$ being the characters of the string $s_i$ in reverse order.

\vspace{8pt}
\hrule

\textbf{Example}

\begin{table}[h]
    \centering
    \begin{tabular}{|p{0.4\linewidth}|p{0.4\linewidth}|}
        \hline
        Input & Output \\
        \hline
        \texttt{\textbf{3}\newline 
        hello\newline 
        goodbye\newline 
        helloagain} & 
        \texttt{olleh\newline
        eybdoog\newline
        niagaolleh} \\
        \hline
    \end{tabular}
\end{table}

\newpage

\addcontentsline{toc}{subsection}{Problem Won}
\LARGE \circled{1} \textbf{Problem Won} \normalsize

{\itshape Are you the quiz champion?}

The Computing Society has had enough. 
Tired of marking all our quiz nights by hand, we've come up with a brilliant new strategy - creating a bespoke scoring application from scratch! 
Excel? 
Google Sheets? 
Never heard of it.
It simultaneously uses 50 conflicting web development frameworks for maximum customisability. 
But in our haste, we forgot to implement the scoring part.
Fancy helping out?

\begin{table}[h]
    \centering
    \begin{tabular}{|c|c|c|c|c|c|}
        \hline
        \textbf{Name} & Round 1 & Round 2 & Round 3 & Round 4 & \textbf{Rounds Won} \\
        \hline
        Alice & 8 & 4 & 1 & \cellcolor{cyan}9 & 1\\
        Bob & \cellcolor{cyan}9 & 4 & \cellcolor{cyan}2 & 3 & \cellcolor{yellow}2 \\
        Charlie & 7 & \cellcolor{cyan}10 & 0 & 5 & 1 \\
        \hline
    \end{tabular}
\end{table}

As you can see by the example quiz above, we just want to find out which player won the most rounds in the quiz, 
assuming only one player gets the highest score each round.

\vspace{8pt}
\hrule

\textbf{Input}

The first line of the input contains two space-separated integers $p$ and $r$ 
- denoting the number of players and number of quiz rounds respectively.

The next $p$ input lines consist of the following space-separated info about each player:
\begin{itemize}
    \item $n$, the player's name
    \item $r$ integers ($s_1, s_2, \dots, s_r$) representing the player's score for each round.
\end{itemize}

\textbf{Constraints}
\begin{equation*}
    1 \leq p \leq 10^3 \qquad 
    1 \leq r \leq 10^4 \qquad 
    1 \leq |n| \leq 11 \qquad 
    \forall i \in \{1, \dots, r\} \quad 0 \leq s_i \leq 10^2
\end{equation*}

Players' names and \textbf{total score}, along with each round's highest score, are always unique.

\textbf{Output}

A single line, containing the name of the player who won the quiz:
the person who won the most rounds!
If multiple players have won the same number of rounds, 
then the tie is broken by the player with the \textit{highest total score} over the course of the quiz.

\vspace{8pt}
\hrule

\textbf{Example}

\begin{table}[h]
    \centering
    \begin{tabular}{|p{0.4\linewidth}|p{0.4\linewidth}|}
        \hline
        Input & Output \\
        \hline
        \textbf{3 5} \newline John 100 5 0 7 4 \newline Barry 63 100 0 6 5 \newline Karen 0 9 1 6 9 & 
        John \\
        \hline
    \end{tabular}
\end{table}

In this example, there are 3 players and 5 rounds in the quiz.
John wins round 1 and round 4. Barry wins round 2. Karen wins round 3 and round 5.
Both John and Karen are tied on the number of rounds won, so we look at their total scores.
John has a total score of 116 compared to Karen's 25, so wins overall. We output John.
\newpage
\addcontentsline{toc}{subsection}{Problem Won}
\LARGE \circled{1} \textbf{Problem Won} \normalsize

{\itshape Are you the quiz champion?}

The Computing Society has had enough. 
Tired of marking all our quiz nights by hand, we've come up with a brilliant new strategy - creating a bespoke scoring application from scratch! 
Excel? 
Google Sheets? 
Never heard of it.
It simultaneously uses 50 conflicting web development frameworks for maximum customisability. 
But in our haste, we forgot to implement the scoring part.
Fancy helping out?

\begin{table}[h]
    \centering
    \begin{tabular}{|c|c|c|c|c|c|}
        \hline
        \textbf{Name} & Round 1 & Round 2 & Round 3 & Round 4 & \textbf{Rounds Won} \\
        \hline
        Alice & 8 & 4 & 1 & \cellcolor{cyan}9 & 1\\
        Bob & \cellcolor{cyan}9 & 4 & \cellcolor{cyan}2 & 3 & \cellcolor{yellow}2 \\
        Charlie & 7 & \cellcolor{cyan}10 & 0 & 5 & 1 \\
        \hline
    \end{tabular}
\end{table}

As you can see by the example quiz above, we just want to find out which player won the most rounds in the quiz, 
assuming only one player gets the highest score each round.

\vspace{8pt}
\hrule

\textbf{Input}

The first line of the input contains two space-separated integers $p$ and $r$ 
- denoting the number of players and number of quiz rounds respectively.

The next $p$ input lines consist of the following space-separated info about each player:
\begin{itemize}
    \item $n$, the player's name
    \item $r$ integers ($s_1, s_2, \dots, s_r$) representing the player's score for each round.
\end{itemize}

\textbf{Constraints}
\begin{equation*}
    1 \leq p \leq 10^3 \qquad 
    1 \leq r \leq 10^4 \qquad 
    1 \leq |n| \leq 11 \qquad 
    \forall i \in \{1, \dots, r\} \quad 0 \leq s_i \leq 10^2
\end{equation*}

Players' names and \textbf{total score}, along with each round's highest score, are always unique.

\textbf{Output}

A single line, containing the name of the player who won the quiz:
the person who won the most rounds!
If multiple players have won the same number of rounds, 
then the tie is broken by the player with the \textit{highest total score} over the course of the quiz.

\vspace{8pt}
\hrule

\textbf{Example}

\begin{table}[h]
    \centering
    \begin{tabular}{|p{0.4\linewidth}|p{0.4\linewidth}|}
        \hline
        Input & Output \\
        \hline
        \textbf{3 5} \newline John 100 5 0 7 4 \newline Barry 63 100 0 6 5 \newline Karen 0 9 1 6 9 & 
        John \\
        \hline
    \end{tabular}
\end{table}

In this example, there are 3 players and 5 rounds in the quiz.
John wins round 1 and round 4. Barry wins round 2. Karen wins round 3 and round 5.
Both John and Karen are tied on the number of rounds won, so we look at their total scores.
John has a total score of 116 compared to Karen's 25, so wins overall. We output John.
\newpage
\addcontentsline{toc}{subsection}{Problem Won}
\LARGE \circled{1} \textbf{Problem Won} \normalsize

{\itshape Are you the quiz champion?}

The Computing Society has had enough. 
Tired of marking all our quiz nights by hand, we've come up with a brilliant new strategy - creating a bespoke scoring application from scratch! 
Excel? 
Google Sheets? 
Never heard of it.
It simultaneously uses 50 conflicting web development frameworks for maximum customisability. 
But in our haste, we forgot to implement the scoring part.
Fancy helping out?

\begin{table}[h]
    \centering
    \begin{tabular}{|c|c|c|c|c|c|}
        \hline
        \textbf{Name} & Round 1 & Round 2 & Round 3 & Round 4 & \textbf{Rounds Won} \\
        \hline
        Alice & 8 & 4 & 1 & \cellcolor{cyan}9 & 1\\
        Bob & \cellcolor{cyan}9 & 4 & \cellcolor{cyan}2 & 3 & \cellcolor{yellow}2 \\
        Charlie & 7 & \cellcolor{cyan}10 & 0 & 5 & 1 \\
        \hline
    \end{tabular}
\end{table}

As you can see by the example quiz above, we just want to find out which player won the most rounds in the quiz, 
assuming only one player gets the highest score each round.

\vspace{8pt}
\hrule

\textbf{Input}

The first line of the input contains two space-separated integers $p$ and $r$ 
- denoting the number of players and number of quiz rounds respectively.

The next $p$ input lines consist of the following space-separated info about each player:
\begin{itemize}
    \item $n$, the player's name
    \item $r$ integers ($s_1, s_2, \dots, s_r$) representing the player's score for each round.
\end{itemize}

\textbf{Constraints}
\begin{equation*}
    1 \leq p \leq 10^3 \qquad 
    1 \leq r \leq 10^4 \qquad 
    1 \leq |n| \leq 11 \qquad 
    \forall i \in \{1, \dots, r\} \quad 0 \leq s_i \leq 10^2
\end{equation*}

Players' names and \textbf{total score}, along with each round's highest score, are always unique.

\textbf{Output}

A single line, containing the name of the player who won the quiz:
the person who won the most rounds!
If multiple players have won the same number of rounds, 
then the tie is broken by the player with the \textit{highest total score} over the course of the quiz.

\vspace{8pt}
\hrule

\textbf{Example}

\begin{table}[h]
    \centering
    \begin{tabular}{|p{0.4\linewidth}|p{0.4\linewidth}|}
        \hline
        Input & Output \\
        \hline
        \textbf{3 5} \newline John 100 5 0 7 4 \newline Barry 63 100 0 6 5 \newline Karen 0 9 1 6 9 & 
        John \\
        \hline
    \end{tabular}
\end{table}

In this example, there are 3 players and 5 rounds in the quiz.
John wins round 1 and round 4. Barry wins round 2. Karen wins round 3 and round 5.
Both John and Karen are tied on the number of rounds won, so we look at their total scores.
John has a total score of 116 compared to Karen's 25, so wins overall. We output John.
\newpage
\addcontentsline{toc}{subsection}{Problem Won}
\LARGE \circled{1} \textbf{Problem Won} \normalsize

{\itshape Are you the quiz champion?}

The Computing Society has had enough. 
Tired of marking all our quiz nights by hand, we've come up with a brilliant new strategy - creating a bespoke scoring application from scratch! 
Excel? 
Google Sheets? 
Never heard of it.
It simultaneously uses 50 conflicting web development frameworks for maximum customisability. 
But in our haste, we forgot to implement the scoring part.
Fancy helping out?

\begin{table}[h]
    \centering
    \begin{tabular}{|c|c|c|c|c|c|}
        \hline
        \textbf{Name} & Round 1 & Round 2 & Round 3 & Round 4 & \textbf{Rounds Won} \\
        \hline
        Alice & 8 & 4 & 1 & \cellcolor{cyan}9 & 1\\
        Bob & \cellcolor{cyan}9 & 4 & \cellcolor{cyan}2 & 3 & \cellcolor{yellow}2 \\
        Charlie & 7 & \cellcolor{cyan}10 & 0 & 5 & 1 \\
        \hline
    \end{tabular}
\end{table}

As you can see by the example quiz above, we just want to find out which player won the most rounds in the quiz, 
assuming only one player gets the highest score each round.

\vspace{8pt}
\hrule

\textbf{Input}

The first line of the input contains two space-separated integers $p$ and $r$ 
- denoting the number of players and number of quiz rounds respectively.

The next $p$ input lines consist of the following space-separated info about each player:
\begin{itemize}
    \item $n$, the player's name
    \item $r$ integers ($s_1, s_2, \dots, s_r$) representing the player's score for each round.
\end{itemize}

\textbf{Constraints}
\begin{equation*}
    1 \leq p \leq 10^3 \qquad 
    1 \leq r \leq 10^4 \qquad 
    1 \leq |n| \leq 11 \qquad 
    \forall i \in \{1, \dots, r\} \quad 0 \leq s_i \leq 10^2
\end{equation*}

Players' names and \textbf{total score}, along with each round's highest score, are always unique.

\textbf{Output}

A single line, containing the name of the player who won the quiz:
the person who won the most rounds!
If multiple players have won the same number of rounds, 
then the tie is broken by the player with the \textit{highest total score} over the course of the quiz.

\vspace{8pt}
\hrule

\textbf{Example}

\begin{table}[h]
    \centering
    \begin{tabular}{|p{0.4\linewidth}|p{0.4\linewidth}|}
        \hline
        Input & Output \\
        \hline
        \textbf{3 5} \newline John 100 5 0 7 4 \newline Barry 63 100 0 6 5 \newline Karen 0 9 1 6 9 & 
        John \\
        \hline
    \end{tabular}
\end{table}

In this example, there are 3 players and 5 rounds in the quiz.
John wins round 1 and round 4. Barry wins round 2. Karen wins round 3 and round 5.
Both John and Karen are tied on the number of rounds won, so we look at their total scores.
John has a total score of 116 compared to Karen's 25, so wins overall. We output John.
\newpage
\addcontentsline{toc}{subsection}{Problem Won}
\LARGE \circled{1} \textbf{Problem Won} \normalsize

{\itshape Are you the quiz champion?}

The Computing Society has had enough. 
Tired of marking all our quiz nights by hand, we've come up with a brilliant new strategy - creating a bespoke scoring application from scratch! 
Excel? 
Google Sheets? 
Never heard of it.
It simultaneously uses 50 conflicting web development frameworks for maximum customisability. 
But in our haste, we forgot to implement the scoring part.
Fancy helping out?

\begin{table}[h]
    \centering
    \begin{tabular}{|c|c|c|c|c|c|}
        \hline
        \textbf{Name} & Round 1 & Round 2 & Round 3 & Round 4 & \textbf{Rounds Won} \\
        \hline
        Alice & 8 & 4 & 1 & \cellcolor{cyan}9 & 1\\
        Bob & \cellcolor{cyan}9 & 4 & \cellcolor{cyan}2 & 3 & \cellcolor{yellow}2 \\
        Charlie & 7 & \cellcolor{cyan}10 & 0 & 5 & 1 \\
        \hline
    \end{tabular}
\end{table}

As you can see by the example quiz above, we just want to find out which player won the most rounds in the quiz, 
assuming only one player gets the highest score each round.

\vspace{8pt}
\hrule

\textbf{Input}

The first line of the input contains two space-separated integers $p$ and $r$ 
- denoting the number of players and number of quiz rounds respectively.

The next $p$ input lines consist of the following space-separated info about each player:
\begin{itemize}
    \item $n$, the player's name
    \item $r$ integers ($s_1, s_2, \dots, s_r$) representing the player's score for each round.
\end{itemize}

\textbf{Constraints}
\begin{equation*}
    1 \leq p \leq 10^3 \qquad 
    1 \leq r \leq 10^4 \qquad 
    1 \leq |n| \leq 11 \qquad 
    \forall i \in \{1, \dots, r\} \quad 0 \leq s_i \leq 10^2
\end{equation*}

Players' names and \textbf{total score}, along with each round's highest score, are always unique.

\textbf{Output}

A single line, containing the name of the player who won the quiz:
the person who won the most rounds!
If multiple players have won the same number of rounds, 
then the tie is broken by the player with the \textit{highest total score} over the course of the quiz.

\vspace{8pt}
\hrule

\textbf{Example}

\begin{table}[h]
    \centering
    \begin{tabular}{|p{0.4\linewidth}|p{0.4\linewidth}|}
        \hline
        Input & Output \\
        \hline
        \textbf{3 5} \newline John 100 5 0 7 4 \newline Barry 63 100 0 6 5 \newline Karen 0 9 1 6 9 & 
        John \\
        \hline
    \end{tabular}
\end{table}

In this example, there are 3 players and 5 rounds in the quiz.
John wins round 1 and round 4. Barry wins round 2. Karen wins round 3 and round 5.
Both John and Karen are tied on the number of rounds won, so we look at their total scores.
John has a total score of 116 compared to Karen's 25, so wins overall. We output John.

\end{document}