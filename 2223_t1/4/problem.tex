\addcontentsline{toc}{subsection}{Catch My Drift}
\LARGE \circled{4} \textbf{Catch My Drift} \normalsize

{\itshape Slowly drifting\dots}

In the real world, clocks aren't perfect. 
Not even digital ones!
They're going to be either a tiny bit slower or faster than the globally agreed upon time, known as a drift rate.
Internal computer networks that consist of multiple devices may need to log events in the correct order, meaning each device's clock time needs to be close to the others.

We call a collection of clocks are '$\delta$-synchronised' if, for each pair of clocks in the collection, the difference between their times is $\leq \delta$.
Over time, the drift rate could cause this condition to fail - with the clocks needing re-synchronisation to remain accurate.

Given a collection of clocks (both time and drift rate), determine the number of seconds until they are no longer $\delta$-syncronised.
Assume each clock has a constant drift rate: the difference between a simulated clock second and real second.

\vspace{8pt}
\hrule

\textbf{Input}

The first line of the input contains an integer $n$, denoting the number of test cases. \\
There will then be $n$ of the following:
\begin{itemize}
    \item A line containing two space-separated integers:
    \begin{itemize}
        \item $\delta$, an integer representing the synchronisation threshold in milliseconds.
        \item $c$, an integer representing the number of clocks that need to be well-syncronised.
    \end{itemize}
    \item $c$ lines, each consisting of two space-separated integers representing clock $i$:
    \begin{itemize}
        \item $t_i$ is an integer representing the unix time of clock $i$ to the nearest millisecond.
        \item $\rho_i$ is an integer representing the drift rate of clock $i$ in milliseconds.
    \end{itemize}
\end{itemize}

\textbf{Constraints}
\begin{equation*}
    1 \leq n \leq 10^2 \qquad 
    0 \leq \delta \leq 10^6 \qquad
    2 \leq c \leq 10^6 \qquad
    \forall i \in \{1, \dots, c\} \quad
    0 \leq t_i \leq 10\delta \qquad
    |\rho_i| \leq 2\delta
\end{equation*}

\textbf{Output}

$n$ lines each consisting of a single integer, the number of seconds it took for the clocks to no longer be $\delta$-synchronised.
If this occurs in-between two seconds, round up. If this will never occur with the current collection of clocks, output \texttt{-1}.

\vspace{8pt}
\hrule

\textbf{Example}

Continued on next page \dots

\newpage

\begin{table}[h]
    \centering
    \begin{tabular}{|p{0.4\linewidth}|p{0.4\linewidth}|}
        \hline
        Input & Output \\
        \hline
        4 
        \newline \textbf{20 2}
        \newline 2 4
        \newline 19 6
        \newline \textbf{500 3} 
        \newline 1000 50 
        \newline 1500 -5 
        \newline 1100 -30 
        \newline \textbf{50 2} 
        \newline 100 -5 
        \newline 120 -5 
        \newline \textbf{2000 3} 
        \newline 3500 700
        \newline 6000 -500
        \newline 5000 0
        &  
        2 \newline 5 \newline -1 \newline 0 \\
        \hline
    \end{tabular}
\end{table}

Let's walk through these four test cases:

\begin{enumerate}
    \item At the start, the two clocks are 17 milliseconds apart from each other. 
    The difference between their drift rates is 2ms, so the time difference will increase to 19ms after one second, and 21ms after two seconds - exceeding the threshold.
    We output 2.
    \item At the start, the first two clocks are on the very edge of the 500ms threshold. 
    However, they then begin to drift towards each other.
    Clocks two and three actually cause the syncronisation to break down, after 5 seconds they are 525ms apart.
    \item These clocks have the same drift rate, and will never drift apart. We output -1.
    \item The first two clocks are already 2500ms apart, so we output 0.
    
\end{enumerate}