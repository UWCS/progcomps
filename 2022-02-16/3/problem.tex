\addcontentsline{toc}{subsection}{Cycles}
\LARGE \textbf{Problem 3 - Cycles} \normalsize

{\itshape Jump around!}

In this problem, consider an array of integers. 
Treat each integer as representing an index in the same array. 
If we follow this trail of indicies along from a specified index, we might return to it - this is called a cycle. 
Consider the array below:

\begin{center}
    \LARGE [2, 4, 1, 5, 0, 3, 7, 10]
\end{center}

The element at index 0 is 2.
So we find the element at index 2, which is 1.
Now we jump to the element at index 1, which is 4.
Then we travel to the element at index 4, which is 0.
Finally, we return back to index 0.
This is a cycle of length 4.

There is also a cycle of length 2 between indices 3 and 5.
There is no cycle starting at index 6, as 10 is outside the index bounds of the array.

Given an array, what is the length of the longest cycle in it?

\vspace{8pt}
\hrule

\textbf{Input}

The first line of the input contains an integer $t$, denoting the number of test cases.

Each of the next $t$ lines contain the space separated integers $n_1, n_2, \dots n_s$, which represent the elements of a zero-indexed array.

\textbf{Constraints}

\begin{itemize}
    \item $1 \leq t \leq 10^5$
    \item $1 \leq s \leq 10^5$
    \item $0 \leq n_i \leq 10^9 \text{ for all } i \in \{1, 2, \dots, s\}$
\end{itemize}

\textbf{Output}

$t$ lines each consisting of a single integer, the length of the array's largest cycle.

If no cycle exists, return 0.

\vspace{8pt}
\hrule

\textbf{Example}

\begin{table}[h]
    \centering
    \begin{tabular}{|p{0.4\linewidth}|p{0.4\linewidth}|}
        \hline
        Input & Output \\
        \hline
        3 \newline 2 4 1 5 0 3 7 10 \newline 5 0 2 1 4 \newline 2 7 3 6 0 3 4 5 & 
        \text{} \newline 4 \newline 1 \newline 5 \\
        \hline
    \end{tabular}
\end{table}

There are 3 test cases, the first of which is explained above, so we output 4.

For the second test, only indices 2 and 4 form self-cycles, so we output 1.

For the final test, there is a single cycle (0, 2, 3, 6, 4), so we output 5.