\addcontentsline{toc}{subsection}{Collision}
\LARGE \textbf{Problem 6 - Collision} \normalsize

{\itshape Could you imagine the PR disaster if people died?}

As a scheduler for Chad Trains, you've been constantly battling with the finance department. 
"More trains, more cheaply, more money!", they say. 
Maybe they don't realise that cutting everything down to a single track wasn't the best idea.

They've sent over the draft timetable for today. 
If approved, there's a pretty high chance people are going to die when the trains collide.
Classic finance.

Let's remove the lethal journeys on here, but try to keep finance happy.

\vspace{8pt}
\hrule

\textbf{Input}

The first line of the input contains an integer $t$, denoting the number of test cases.

For each test case:

\begin{itemize}
    \item The first line will have two space separated integers $n$ and $m$.
    \item The next $n$ lines will each contain a space separated string and integer, 
    \\ representing the station name and distance in km along the track.
    \item The next $m$ lines will contain space-separated journey information: 
    \begin{itemize}
        \item $s_1, s_2$ represent the start and end station names.
        \item $t_1, t_2$ represent the start and end times in 24h format (such as \texttt{13:48})
    \end{itemize}
\end{itemize}

See the example for more inforation.

\textbf{Constraints}

\begin{itemize}
    \item $2 \leq n \leq 50$
    \item $1 \leq m \leq 1000$
    \item $00\text{:}00 \leq t_1 < t_2 \leq 23\text{:}59$  
\end{itemize}

\textbf{Output}

$t$ lines each consisting of a collection of $j$ space-separated integers, 
representing the zero-indexed positions of the journeys which will be left operating for the day.

A collision is where two trains overlap at any point of their journey.

You may assume that each train travels at a constant speed.

Note that trains arriving or leaving at the same station at the same time from opposite directions is not counted as a collision.

\vspace{8pt}
\hrule

\textbf{Scoring}

\begin{itemize}
    \item $-100$ Points if a schedule results in any collision. Minor PR disaster.
    \item $1 + 2 + \dots + j$ Points if a schedule is safe.
\end{itemize}

\vspace{8pt}
\hrule

\textbf{Example}

Continued on next page...

\newpage

\begin{table}[h]
    \centering
    \begin{tabular}{|p{0.5\linewidth}|p{0.3\linewidth}|}
        \hline
        Input & Output \\
        \hline
        2 \newline 
        2 3 \newline
        Istanbul 0 \newline
        Constantinople 99 \newline
        Istanbul Constantinople 03:45 09:49 \newline
        Constantinople Istanbul 11:30 16:43 \newline
        Istanbul Constantinople 12:38 17:52 \newline
        3 5 \newline 
        Madrid 0 \newline 
        Paris 70 \newline 
        Berlin 160 \newline 
        Madrid Berlin 00:05 02:04 \newline 
        Berlin Madrid 01:59 16:48 \newline
        Paris Berlin 08:13 09:51 \newline
        Paris Madrid 08:13 08:36 \newline
        Berlin Paris 09:51 11:17 \newline & 
        0 1 \newline 0 2 3 4 \\
        \hline
    \end{tabular}
\end{table}

There are two test cases here:

\begin{enumerate}
    \item The two stations are Istanbul and Constantinople, distance 99km apart. 
    \\ There are three journeys scheduled. Journey 0 does not overlap any other.
    Journey 1 and 2 overlap, and so one must be removed.
    \item The three stations are Madrid, Paris and Berlin.
    Of the 5 journeys scheduled, only Journey 1 needs to be removed. While journeys 2 and 3 leave the same station at the same time, the trains are travelling in opposite directions.
\end{enumerate}