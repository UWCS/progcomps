\addcontentsline{toc}{subsection}{First Move}
\LARGE \textbf{Problem 4 - First Move} \normalsize

{\itshape A professional game for hardcore sporting fans}

It's the 50th international Tic-Tac-Toe championships, and as head adjudicator, 
the integrity of the sport rests in your hands alone. 
You have been alerted to the fact that some players are not following the rules,
and decide to investigate.

At an international level, the rules have a slight modification. 
Player X and Player O take turns placing their tiles on an $n$ by $n$ board, 
with the aim of getting $k$-in-a-row of their tile, horizontally, vertically, or diagonally. 
Either may start first.

For each game, you need to determine whether the board state could have been legally obtained, 
and if so, try to determine the player who made the first move.

\vspace{8pt}
\hrule

\textbf{Input}

The first line of the input contains an integer $t$, denoting the number of test cases.

For each test case:

\begin{itemize}
    \item The first line consists of two space-separated integers $n$ and $k$.
    \item The next $n$ lines consist of $n$ characters each, representing the board.
\end{itemize}

Tiles on the board may be empty (the \texttt{.} character), but could be an \texttt{X} or \texttt{O} instead.

\textbf{Constraints}

\begin{itemize}
    \item $1 \leq k \leq n \leq 100$
\end{itemize}

\textbf{Output}

$t$ lines each consisting of one of four possible characters. The character chosen depends on the properties of the $k$-in-a-row board:

\begin{itemize}
    \item[\texttt{!}] - The board is invalid.
    \item[\texttt{X}] - Player X must have made the first move.
    \item[\texttt{O}] - Player O must have made the first move.
    \item[\texttt{?}] - It is impossible to determine who made the first move.
\end{itemize}

\vspace{8pt}
\hrule

\textbf{Example}

Continued on next page...

\newpage

\begin{table}[h]
    \centering
    \begin{tabular}{|p{0.4\linewidth}|p{0.4\linewidth}|}
        \hline
        Input & Output \\
        \hline
        {\ttfamily 
        5 \newline 
        2 2 \newline
        .X \newline O. \newline
        3 3 \newline
        ..X \newline X.X \newline X.O \newline
        3 3 \newline
        .XO \newline .XO \newline .XO \newline
        4 3 \newline 
        .O.. \newline ..O. \newline X..O \newline ..X. \newline
        5 3 \newline
        ..... \newline .XXX. \newline O.X.. \newline .X..O \newline .O.O.
        } &  
        {\ttfamily ? \newline ! \newline ! \newline O \newline X} \\
        \hline
    \end{tabular}
\end{table}

There are 5 test cases:

\begin{enumerate}
    \item Each player has a single tile, so it's impossible to determine who went first.
    \item Player X has placed too many tiles on the board out of turn.
    \item It is impossible for both players to win the game.
    \item Player O made the winning move and has more tiles, so must have gone first.
    \item Player X made the winning move and has more tiles, so must have gone first.
\end{enumerate}