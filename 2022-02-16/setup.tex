\section*{Setup Instructions}

To participate, you'll need to download a program that sends your solutions to our server to be marked.
You can download the command-line program from \url{https://github.com/Jlobblet/ProgcompMarker/releases}.

Select the file appropriate to your operating system and architecture.
On a DCS machine, this is \verb|ProgcompCli-linux-x64.zip|.
Download the correct file and then extract it\footnote{You can use {\ttfamily unzip} or {\ttfamily tar -xf} for {\ttfamily .zip} and {\ttfamily .tar.xz} files respectively, or a program such as 7zip or WinRAR.}.

Go into the extracted folder.
Here you will find an executable called \verb|ProgcompCli|\footnote{On Windows, the executable will be called {\ttfamily ProgcompCli.exe}.}.
Run this executable from the command line (\verb|./ProgcompCli|).

The program will prompt you to perform first-time setup.
Follow the instructions on-screen, entering \verb|https://progcomp.uwcs.co.uk| for the endpoint.

To run your solution, provide the problem number and an executable to the CLI: \verb|./ProgcompCli [PROBLEM NUMBER] [PATH TO EXECUTABLE]|.


\subsection*{Notes}

If you are writing an interpreted language (such as Python), it may be necessary to add a \href{https://en.wikipedia.org/wiki/Shebang_(Unix)}{\em shebang} to the entry point file.
For example: \verb|#!/usr/bin/env python3| followed by the rest of the Python script as usual.

Once set up, you can attempt Problem 0 - this is just to read in an input line, and then send it back.
