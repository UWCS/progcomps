\documentclass[a4paper,12pt,parskip=half-]{scrartcl}

% Packages

\usepackage[english]{babel}
\usepackage{array}
\usepackage{amsmath}
\usepackage[colorlinks=true, allcolors=blue]{hyperref}
\usepackage{graphicx}
\usepackage[a4paper,top=2cm,bottom=2cm,left=2cm,right=2cm,marginparwidth=1.75cm]{geometry}

\usepackage[defaultfam,tabular,lining]{montserrat}
\usepackage[scaled]{beramono}
\usepackage[T1]{fontenc}
\renewcommand*\oldstylenums[1]{{\fontfamily{Montserrat-TOsF}\selectfont #1}}
\renewcommand*{\ttdefault}{cmr}


\usepackage{wrapfig}
\usepackage{pmboxdraw}
\usepackage[most]{tcolorbox}

\graphicspath{{../assets/}}
\pagestyle{empty}
\setkomafont{section}{\usefont{T1}{fvs}{b}{n}\Large}

\renewcommand{\ttdefault}{ascii}
\renewcommand{\arraystretch}{1.5}

\newtcolorbox{exampleinput}[1][]{
    sharp corners,
    enhanced,
    colback=white,
    attach title to upper,
    #1
}

\begin{document}

\begin{wrapfigure}{r}{0.2\textwidth}
    \vspace{-10pt} % Move shield in line with text
    \includegraphics[width=0.8\linewidth]{shield.png} % Replace with trophy with dots
    \vspace{-100pt} % Prevent text below from moving out of the way
\end{wrapfigure}

\normalfont \Huge \bfseries UWCS Challenge++

\normalfont\Large A friendly programming competition
\normalsize

\section*{Introduction}

Welcome! 
Inside this document, you'll find specifications for each the 6 problems used at the event held on 16th February, 2022. Note that in each problem, the number of test cases $t$ is exactly 10000, with inputs of varying difficulties used.

\tableofcontents

\newpage

\addcontentsline{toc}{section}{Problems}

\addcontentsline{toc}{subsection}{Problem Won}
\LARGE \circled{1} \textbf{Problem Won} \normalsize

{\itshape Are you the quiz champion?}

The Computing Society has had enough. 
Tired of marking all our quiz nights by hand, we've come up with a brilliant new strategy - creating a bespoke scoring application from scratch! 
Excel? 
Google Sheets? 
Never heard of it.
It simultaneously uses 50 conflicting web development frameworks for maximum customisability. 
But in our haste, we forgot to implement the scoring part.
Fancy helping out?

\begin{table}[h]
    \centering
    \begin{tabular}{|c|c|c|c|c|c|}
        \hline
        \textbf{Name} & Round 1 & Round 2 & Round 3 & Round 4 & \textbf{Rounds Won} \\
        \hline
        Alice & 8 & 4 & 1 & \cellcolor{cyan}9 & 1\\
        Bob & \cellcolor{cyan}9 & 4 & \cellcolor{cyan}2 & 3 & \cellcolor{yellow}2 \\
        Charlie & 7 & \cellcolor{cyan}10 & 0 & 5 & 1 \\
        \hline
    \end{tabular}
\end{table}

As you can see by the example quiz above, we just want to find out which player won the most rounds in the quiz, 
assuming only one player gets the highest score each round.

\vspace{8pt}
\hrule

\textbf{Input}

The first line of the input contains two space-separated integers $p$ and $r$ 
- denoting the number of players and number of quiz rounds respectively.

The next $p$ input lines consist of the following space-separated info about each player:
\begin{itemize}
    \item $n$, the player's name
    \item $r$ integers ($s_1, s_2, \dots, s_r$) representing the player's score for each round.
\end{itemize}

\textbf{Constraints}
\begin{equation*}
    1 \leq p \leq 10^3 \qquad 
    1 \leq r \leq 10^4 \qquad 
    1 \leq |n| \leq 11 \qquad 
    \forall i \in \{1, \dots, r\} \quad 0 \leq s_i \leq 10^2
\end{equation*}

Players' names and \textbf{total score}, along with each round's highest score, are always unique.

\textbf{Output}

A single line, containing the name of the player who won the quiz:
the person who won the most rounds!
If multiple players have won the same number of rounds, 
then the tie is broken by the player with the \textit{highest total score} over the course of the quiz.

\vspace{8pt}
\hrule

\textbf{Example}

\begin{table}[h]
    \centering
    \begin{tabular}{|p{0.4\linewidth}|p{0.4\linewidth}|}
        \hline
        Input & Output \\
        \hline
        \textbf{3 5} \newline John 100 5 0 7 4 \newline Barry 63 100 0 6 5 \newline Karen 0 9 1 6 9 & 
        John \\
        \hline
    \end{tabular}
\end{table}

In this example, there are 3 players and 5 rounds in the quiz.
John wins round 1 and round 4. Barry wins round 2. Karen wins round 3 and round 5.
Both John and Karen are tied on the number of rounds won, so we look at their total scores.
John has a total score of 116 compared to Karen's 25, so wins overall. We output John.
\newpage
\addcontentsline{toc}{subsection}{Problem Won}
\LARGE \circled{1} \textbf{Problem Won} \normalsize

{\itshape Are you the quiz champion?}

The Computing Society has had enough. 
Tired of marking all our quiz nights by hand, we've come up with a brilliant new strategy - creating a bespoke scoring application from scratch! 
Excel? 
Google Sheets? 
Never heard of it.
It simultaneously uses 50 conflicting web development frameworks for maximum customisability. 
But in our haste, we forgot to implement the scoring part.
Fancy helping out?

\begin{table}[h]
    \centering
    \begin{tabular}{|c|c|c|c|c|c|}
        \hline
        \textbf{Name} & Round 1 & Round 2 & Round 3 & Round 4 & \textbf{Rounds Won} \\
        \hline
        Alice & 8 & 4 & 1 & \cellcolor{cyan}9 & 1\\
        Bob & \cellcolor{cyan}9 & 4 & \cellcolor{cyan}2 & 3 & \cellcolor{yellow}2 \\
        Charlie & 7 & \cellcolor{cyan}10 & 0 & 5 & 1 \\
        \hline
    \end{tabular}
\end{table}

As you can see by the example quiz above, we just want to find out which player won the most rounds in the quiz, 
assuming only one player gets the highest score each round.

\vspace{8pt}
\hrule

\textbf{Input}

The first line of the input contains two space-separated integers $p$ and $r$ 
- denoting the number of players and number of quiz rounds respectively.

The next $p$ input lines consist of the following space-separated info about each player:
\begin{itemize}
    \item $n$, the player's name
    \item $r$ integers ($s_1, s_2, \dots, s_r$) representing the player's score for each round.
\end{itemize}

\textbf{Constraints}
\begin{equation*}
    1 \leq p \leq 10^3 \qquad 
    1 \leq r \leq 10^4 \qquad 
    1 \leq |n| \leq 11 \qquad 
    \forall i \in \{1, \dots, r\} \quad 0 \leq s_i \leq 10^2
\end{equation*}

Players' names and \textbf{total score}, along with each round's highest score, are always unique.

\textbf{Output}

A single line, containing the name of the player who won the quiz:
the person who won the most rounds!
If multiple players have won the same number of rounds, 
then the tie is broken by the player with the \textit{highest total score} over the course of the quiz.

\vspace{8pt}
\hrule

\textbf{Example}

\begin{table}[h]
    \centering
    \begin{tabular}{|p{0.4\linewidth}|p{0.4\linewidth}|}
        \hline
        Input & Output \\
        \hline
        \textbf{3 5} \newline John 100 5 0 7 4 \newline Barry 63 100 0 6 5 \newline Karen 0 9 1 6 9 & 
        John \\
        \hline
    \end{tabular}
\end{table}

In this example, there are 3 players and 5 rounds in the quiz.
John wins round 1 and round 4. Barry wins round 2. Karen wins round 3 and round 5.
Both John and Karen are tied on the number of rounds won, so we look at their total scores.
John has a total score of 116 compared to Karen's 25, so wins overall. We output John.
\newpage
\addcontentsline{toc}{subsection}{Problem Won}
\LARGE \circled{1} \textbf{Problem Won} \normalsize

{\itshape Are you the quiz champion?}

The Computing Society has had enough. 
Tired of marking all our quiz nights by hand, we've come up with a brilliant new strategy - creating a bespoke scoring application from scratch! 
Excel? 
Google Sheets? 
Never heard of it.
It simultaneously uses 50 conflicting web development frameworks for maximum customisability. 
But in our haste, we forgot to implement the scoring part.
Fancy helping out?

\begin{table}[h]
    \centering
    \begin{tabular}{|c|c|c|c|c|c|}
        \hline
        \textbf{Name} & Round 1 & Round 2 & Round 3 & Round 4 & \textbf{Rounds Won} \\
        \hline
        Alice & 8 & 4 & 1 & \cellcolor{cyan}9 & 1\\
        Bob & \cellcolor{cyan}9 & 4 & \cellcolor{cyan}2 & 3 & \cellcolor{yellow}2 \\
        Charlie & 7 & \cellcolor{cyan}10 & 0 & 5 & 1 \\
        \hline
    \end{tabular}
\end{table}

As you can see by the example quiz above, we just want to find out which player won the most rounds in the quiz, 
assuming only one player gets the highest score each round.

\vspace{8pt}
\hrule

\textbf{Input}

The first line of the input contains two space-separated integers $p$ and $r$ 
- denoting the number of players and number of quiz rounds respectively.

The next $p$ input lines consist of the following space-separated info about each player:
\begin{itemize}
    \item $n$, the player's name
    \item $r$ integers ($s_1, s_2, \dots, s_r$) representing the player's score for each round.
\end{itemize}

\textbf{Constraints}
\begin{equation*}
    1 \leq p \leq 10^3 \qquad 
    1 \leq r \leq 10^4 \qquad 
    1 \leq |n| \leq 11 \qquad 
    \forall i \in \{1, \dots, r\} \quad 0 \leq s_i \leq 10^2
\end{equation*}

Players' names and \textbf{total score}, along with each round's highest score, are always unique.

\textbf{Output}

A single line, containing the name of the player who won the quiz:
the person who won the most rounds!
If multiple players have won the same number of rounds, 
then the tie is broken by the player with the \textit{highest total score} over the course of the quiz.

\vspace{8pt}
\hrule

\textbf{Example}

\begin{table}[h]
    \centering
    \begin{tabular}{|p{0.4\linewidth}|p{0.4\linewidth}|}
        \hline
        Input & Output \\
        \hline
        \textbf{3 5} \newline John 100 5 0 7 4 \newline Barry 63 100 0 6 5 \newline Karen 0 9 1 6 9 & 
        John \\
        \hline
    \end{tabular}
\end{table}

In this example, there are 3 players and 5 rounds in the quiz.
John wins round 1 and round 4. Barry wins round 2. Karen wins round 3 and round 5.
Both John and Karen are tied on the number of rounds won, so we look at their total scores.
John has a total score of 116 compared to Karen's 25, so wins overall. We output John.
% \newpage
% \addcontentsline{toc}{subsection}{Problem Won}
\LARGE \circled{1} \textbf{Problem Won} \normalsize

{\itshape Are you the quiz champion?}

The Computing Society has had enough. 
Tired of marking all our quiz nights by hand, we've come up with a brilliant new strategy - creating a bespoke scoring application from scratch! 
Excel? 
Google Sheets? 
Never heard of it.
It simultaneously uses 50 conflicting web development frameworks for maximum customisability. 
But in our haste, we forgot to implement the scoring part.
Fancy helping out?

\begin{table}[h]
    \centering
    \begin{tabular}{|c|c|c|c|c|c|}
        \hline
        \textbf{Name} & Round 1 & Round 2 & Round 3 & Round 4 & \textbf{Rounds Won} \\
        \hline
        Alice & 8 & 4 & 1 & \cellcolor{cyan}9 & 1\\
        Bob & \cellcolor{cyan}9 & 4 & \cellcolor{cyan}2 & 3 & \cellcolor{yellow}2 \\
        Charlie & 7 & \cellcolor{cyan}10 & 0 & 5 & 1 \\
        \hline
    \end{tabular}
\end{table}

As you can see by the example quiz above, we just want to find out which player won the most rounds in the quiz, 
assuming only one player gets the highest score each round.

\vspace{8pt}
\hrule

\textbf{Input}

The first line of the input contains two space-separated integers $p$ and $r$ 
- denoting the number of players and number of quiz rounds respectively.

The next $p$ input lines consist of the following space-separated info about each player:
\begin{itemize}
    \item $n$, the player's name
    \item $r$ integers ($s_1, s_2, \dots, s_r$) representing the player's score for each round.
\end{itemize}

\textbf{Constraints}
\begin{equation*}
    1 \leq p \leq 10^3 \qquad 
    1 \leq r \leq 10^4 \qquad 
    1 \leq |n| \leq 11 \qquad 
    \forall i \in \{1, \dots, r\} \quad 0 \leq s_i \leq 10^2
\end{equation*}

Players' names and \textbf{total score}, along with each round's highest score, are always unique.

\textbf{Output}

A single line, containing the name of the player who won the quiz:
the person who won the most rounds!
If multiple players have won the same number of rounds, 
then the tie is broken by the player with the \textit{highest total score} over the course of the quiz.

\vspace{8pt}
\hrule

\textbf{Example}

\begin{table}[h]
    \centering
    \begin{tabular}{|p{0.4\linewidth}|p{0.4\linewidth}|}
        \hline
        Input & Output \\
        \hline
        \textbf{3 5} \newline John 100 5 0 7 4 \newline Barry 63 100 0 6 5 \newline Karen 0 9 1 6 9 & 
        John \\
        \hline
    \end{tabular}
\end{table}

In this example, there are 3 players and 5 rounds in the quiz.
John wins round 1 and round 4. Barry wins round 2. Karen wins round 3 and round 5.
Both John and Karen are tied on the number of rounds won, so we look at their total scores.
John has a total score of 116 compared to Karen's 25, so wins overall. We output John.
% \newpage
% \addcontentsline{toc}{subsection}{Problem Won}
\LARGE \circled{1} \textbf{Problem Won} \normalsize

{\itshape Are you the quiz champion?}

The Computing Society has had enough. 
Tired of marking all our quiz nights by hand, we've come up with a brilliant new strategy - creating a bespoke scoring application from scratch! 
Excel? 
Google Sheets? 
Never heard of it.
It simultaneously uses 50 conflicting web development frameworks for maximum customisability. 
But in our haste, we forgot to implement the scoring part.
Fancy helping out?

\begin{table}[h]
    \centering
    \begin{tabular}{|c|c|c|c|c|c|}
        \hline
        \textbf{Name} & Round 1 & Round 2 & Round 3 & Round 4 & \textbf{Rounds Won} \\
        \hline
        Alice & 8 & 4 & 1 & \cellcolor{cyan}9 & 1\\
        Bob & \cellcolor{cyan}9 & 4 & \cellcolor{cyan}2 & 3 & \cellcolor{yellow}2 \\
        Charlie & 7 & \cellcolor{cyan}10 & 0 & 5 & 1 \\
        \hline
    \end{tabular}
\end{table}

As you can see by the example quiz above, we just want to find out which player won the most rounds in the quiz, 
assuming only one player gets the highest score each round.

\vspace{8pt}
\hrule

\textbf{Input}

The first line of the input contains two space-separated integers $p$ and $r$ 
- denoting the number of players and number of quiz rounds respectively.

The next $p$ input lines consist of the following space-separated info about each player:
\begin{itemize}
    \item $n$, the player's name
    \item $r$ integers ($s_1, s_2, \dots, s_r$) representing the player's score for each round.
\end{itemize}

\textbf{Constraints}
\begin{equation*}
    1 \leq p \leq 10^3 \qquad 
    1 \leq r \leq 10^4 \qquad 
    1 \leq |n| \leq 11 \qquad 
    \forall i \in \{1, \dots, r\} \quad 0 \leq s_i \leq 10^2
\end{equation*}

Players' names and \textbf{total score}, along with each round's highest score, are always unique.

\textbf{Output}

A single line, containing the name of the player who won the quiz:
the person who won the most rounds!
If multiple players have won the same number of rounds, 
then the tie is broken by the player with the \textit{highest total score} over the course of the quiz.

\vspace{8pt}
\hrule

\textbf{Example}

\begin{table}[h]
    \centering
    \begin{tabular}{|p{0.4\linewidth}|p{0.4\linewidth}|}
        \hline
        Input & Output \\
        \hline
        \textbf{3 5} \newline John 100 5 0 7 4 \newline Barry 63 100 0 6 5 \newline Karen 0 9 1 6 9 & 
        John \\
        \hline
    \end{tabular}
\end{table}

In this example, there are 3 players and 5 rounds in the quiz.
John wins round 1 and round 4. Barry wins round 2. Karen wins round 3 and round 5.
Both John and Karen are tied on the number of rounds won, so we look at their total scores.
John has a total score of 116 compared to Karen's 25, so wins overall. We output John.
% \newpage
% \addcontentsline{toc}{subsection}{Problem Won}
\LARGE \circled{1} \textbf{Problem Won} \normalsize

{\itshape Are you the quiz champion?}

The Computing Society has had enough. 
Tired of marking all our quiz nights by hand, we've come up with a brilliant new strategy - creating a bespoke scoring application from scratch! 
Excel? 
Google Sheets? 
Never heard of it.
It simultaneously uses 50 conflicting web development frameworks for maximum customisability. 
But in our haste, we forgot to implement the scoring part.
Fancy helping out?

\begin{table}[h]
    \centering
    \begin{tabular}{|c|c|c|c|c|c|}
        \hline
        \textbf{Name} & Round 1 & Round 2 & Round 3 & Round 4 & \textbf{Rounds Won} \\
        \hline
        Alice & 8 & 4 & 1 & \cellcolor{cyan}9 & 1\\
        Bob & \cellcolor{cyan}9 & 4 & \cellcolor{cyan}2 & 3 & \cellcolor{yellow}2 \\
        Charlie & 7 & \cellcolor{cyan}10 & 0 & 5 & 1 \\
        \hline
    \end{tabular}
\end{table}

As you can see by the example quiz above, we just want to find out which player won the most rounds in the quiz, 
assuming only one player gets the highest score each round.

\vspace{8pt}
\hrule

\textbf{Input}

The first line of the input contains two space-separated integers $p$ and $r$ 
- denoting the number of players and number of quiz rounds respectively.

The next $p$ input lines consist of the following space-separated info about each player:
\begin{itemize}
    \item $n$, the player's name
    \item $r$ integers ($s_1, s_2, \dots, s_r$) representing the player's score for each round.
\end{itemize}

\textbf{Constraints}
\begin{equation*}
    1 \leq p \leq 10^3 \qquad 
    1 \leq r \leq 10^4 \qquad 
    1 \leq |n| \leq 11 \qquad 
    \forall i \in \{1, \dots, r\} \quad 0 \leq s_i \leq 10^2
\end{equation*}

Players' names and \textbf{total score}, along with each round's highest score, are always unique.

\textbf{Output}

A single line, containing the name of the player who won the quiz:
the person who won the most rounds!
If multiple players have won the same number of rounds, 
then the tie is broken by the player with the \textit{highest total score} over the course of the quiz.

\vspace{8pt}
\hrule

\textbf{Example}

\begin{table}[h]
    \centering
    \begin{tabular}{|p{0.4\linewidth}|p{0.4\linewidth}|}
        \hline
        Input & Output \\
        \hline
        \textbf{3 5} \newline John 100 5 0 7 4 \newline Barry 63 100 0 6 5 \newline Karen 0 9 1 6 9 & 
        John \\
        \hline
    \end{tabular}
\end{table}

In this example, there are 3 players and 5 rounds in the quiz.
John wins round 1 and round 4. Barry wins round 2. Karen wins round 3 and round 5.
Both John and Karen are tied on the number of rounds won, so we look at their total scores.
John has a total score of 116 compared to Karen's 25, so wins overall. We output John.

% \newpage

% \begin{center}
%     \Huge \textbf{Ahead be spoilers!} \normalsize
% \end{center}

% \newpage

% \addcontentsline{toc}{section}{Discussion (Coming Soon)}

% % \addcontentsline{toc}{subsection}{Persistence}
% \LARGE \textbf{Persistence} \normalsize

% This was intended to be the simplest problem. 
% We saw two main approaches used to tackle it:

% \begin{itemize}
%     \item M A T H S
%     \item Type conversions between strings and integers
% \end{itemize}
% \newpage
% % \addcontentsline{toc}{subsection}{Persistence}
% \LARGE \textbf{Persistence} \normalsize

% This was intended to be the simplest problem. 
% We saw two main approaches used to tackle it:

% \begin{itemize}
%     \item M A T H S
%     \item Type conversions between strings and integers
% \end{itemize}
% \newpage
% % \addcontentsline{toc}{subsection}{Persistence}
% \LARGE \textbf{Persistence} \normalsize

% This was intended to be the simplest problem. 
% We saw two main approaches used to tackle it:

% \begin{itemize}
%     \item M A T H S
%     \item Type conversions between strings and integers
% \end{itemize}
% \newpage
% % \addcontentsline{toc}{subsection}{Persistence}
% \LARGE \textbf{Persistence} \normalsize

% This was intended to be the simplest problem. 
% We saw two main approaches used to tackle it:

% \begin{itemize}
%     \item M A T H S
%     \item Type conversions between strings and integers
% \end{itemize}
% \newpage
% % \addcontentsline{toc}{subsection}{Persistence}
% \LARGE \textbf{Persistence} \normalsize

% This was intended to be the simplest problem. 
% We saw two main approaches used to tackle it:

% \begin{itemize}
%     \item M A T H S
%     \item Type conversions between strings and integers
% \end{itemize}

\end{document}